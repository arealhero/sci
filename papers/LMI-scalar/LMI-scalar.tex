\documentclass[a4paper,14pt]{article}

\usepackage{pgfpages}

\usepackage[T2A]{fontenc}
\usepackage[utf8]{inputenc}
\usepackage[russian]{babel}

\usepackage{physics}
\usepackage{amsthm, amsmath, amssymb}
\usepackage{mathtext}
\usepackage[makeroom]{cancel}

% biber recommends this
\usepackage{csquotes}

\usepackage[
    colorlinks=true,
    linkcolor=black,
    anchorcolor=black,
    citecolor=black,
    filecolor=black,
    menucolor=black,
    runcolor=black,
    urlcolor=black
]{hyperref}
\usepackage[backend=biber,citestyle=numeric]{biblatex}
\addbibresource{~/data/references.bib}

\usepackage{graphicx}
\usepackage{graphbox}
\graphicspath{ {./images/} }

\linespread{1.5}
\setlength{\parindent}{1.25cm}

\usepackage{geometry}
\geometry{left=3cm}
\geometry{right=1.5cm}
\geometry{top=2cm}
\geometry{bottom=2cm}

\newtheorem{theorem}{Теорема}
% \newtheorem{proposition}{Утверждение}
% \newtheorem{lemma}{Лемма}

% \theoremstyle{definition}
% \newtheorem{definition}{Определение}


\title{Оценка степени затухания и перерегулирования в линейной скалярной системе с запаздыванием}
\author{Шаршуков Владислав}
\date{2023}

\begin{document}

\maketitle

\newpage

\setcounter{page}{2}

\begin{center}
  \tableofcontents
\end{center}

\newpage

\section{Введение}

\section{Постановка задачи}

Рассмотрим линейную скалярную систему с запаздыванием:
\begin{equation}
  \label{eq:main-system}
  \dot x(t) = a_0 x(t) + a_1 x(t - h), \qquad h = \mathrm{const} > 0.
\end{equation}
Будем считать её экспоненциально устойчивой. В этом случае существуют
постоянные $\sigma > 0$ и $\gamma \geqslant 1$ такие, что
для любого решения системы~\eqref{eq:main-system} справедлива
оценка
\begin{equation*}
  \abs{x(t, \varphi)} \leqslant \gamma e^{-\sigma t} \norm{\varphi}_h,
  \qquad
  t > 0,
  \quad
  \varphi \in PC([-h, 0], \mathbb{R}).
\end{equation*}
Также будем считать, что $a_1 \neq 0$. Требуется определить $\gamma$ и $\sigma$
так, чтобы оценка была наилучшей.

\section{Результат}

В работе~\cite{modie2005} было получено условие в виде матричного неравенства,
при котором можно вычислить значения $\gamma$ и $\sigma$. В скалярном случае этот
результат можно сформулировать так:
\begin{theorem}\label{th:modie}
  Если существуют $P > 0, Q > 0$ и $\beta > 0$ такие, что
  справедливо неравенство
  \begin{equation}
    \label{eq:lmi}
    \mathcal{M}(P, Q) + 2 \beta \mathcal{N}(P) \prec 0,
  \end{equation}
  где
  \begin{equation*}
    \begin{aligned}
      \mathcal{M}(P, Q)
      &=
        \begin{bmatrix}
          2 a_0 P + Q & a_1 P \\
          a_1 P & - e^{-2 \beta h} Q
        \end{bmatrix}, \\
      \mathcal{N}(P)
      &=
        \begin{bmatrix}
          P & 0 \\
          0 & 0
        \end{bmatrix},
    \end{aligned}
  \end{equation*}
  тогда
  \begin{equation*}
    \abs{x(t, \varphi)} \leqslant \sqrt{ \frac{\alpha_2}{\alpha_1} } e^{-\beta t} \norm{\varphi}_h,
  \end{equation*}
  где положительные постоянные $\alpha_1$ и $\alpha_2$ определены как
  \begin{equation*}
    \begin{aligned}
      \alpha_1 &= P, \\
      \alpha_2 &= P + h Q.
    \end{aligned}
  \end{equation*}
\end{theorem}

Выясним, при каких условиях матрица
\begin{equation*}
  \mathcal{M}(P, Q) + 2 \beta \mathcal{N}(P)
  =
  \begin{bmatrix}
    2 (a_0 + \beta) P + Q & a_1 P \\
    a_1 P & - e^{-2 \beta h} Q
  \end{bmatrix}
\end{equation*}
отрицательно определена. Применяя критерий Сильвестра, получаем
следующие условия на главные миноры:
\begin{equation*}
  \begin{aligned}
    2 (a_0 + \beta) P + Q &< 0, \\
    \left[
    2 (a_0 + \beta) P + Q
    \right] e^{-2 \beta h} Q + a_1^2 P^2 &< 0.
  \end{aligned}
\end{equation*}
Так как $P > 0$, то
\begin{equation*}
  \begin{aligned}
    2 (a_0 + \beta) + \frac{Q}{P} &< 0, \\
    \left[
    2 (a_0 + \beta) + \frac{Q}{P}
    \right] e^{-2 \beta h} \frac{Q}{P} + a_1^2 &< 0.
  \end{aligned}
\end{equation*}
Введём обозначение: $z = \dfrac{Q}{P}$ (заметим, что $z > 0$). Пользуясь им,
условия можно переписать в виде
\begin{equation*}
  \begin{aligned}
    2 (a_0 + \beta) + z &< 0, \\
    \left[
    2 (a_0 + \beta) + z
    \right] e^{-2 \beta h} z + a_1^2 &< 0.
  \end{aligned}
\end{equation*}

Если $\beta > 0$ задана, то из первого условия следует, что
\begin{equation*}
  0 < z < -2 (a_0 + \beta).
\end{equation*}
Значит,
\begin{equation*}
  a_0 + \beta < 0,
  \implies
  0 < \beta < - a_0,
\end{equation*}
откуда следует, что $a_0 < 0$.

Рассмотрим теперь второе условие:
\begin{equation*}
  \left[
    2 (a_0 + \beta) + z
  \right] e^{-2 \beta h} z + a_1^2
  =
  \left[
    2 (a_0 + \beta) z + z^2
  \right] e^{-2 \beta h} + a_1^2 < 0.
\end{equation*}
Так как $e^{-2 \beta h} > 0$, то его можно переписать в виде
\begin{equation*}
  z^2 + 2 (a_0 + \beta) z + a_1^2 e^{2 \beta h} < 0.
\end{equation*}
Найдём корни этого квадратного трёхчлена. Пусть
\begin{equation*}
  D = 4 {(a_0 + \beta)}^2 - 4 a_1^2 e^{2 \beta h} = 4 D_1,
\end{equation*}
где
\begin{equation*}
  D_1 = {(a_0 + \beta)}^2 - a_1^2 e^{2 \beta h}.
\end{equation*}
Так как $z \in \mathbb{R}$, то $D \geqslant 0$. Найдём такое $\beta \in (0, -a_0)$,
при котором $D = 0$, то есть
\begin{equation}
  \label{eq:D-zero}
  {(a_0 + \beta)}^2 = a_1^2 e^{2 \beta h}.
\end{equation}
На рассматриваемом интервале левая часть~\eqref{eq:D-zero} монотонно убывает, а
правая часть монотонно возрастает. Учитывая, что при $\beta = 0$ левая часть
равна $a_0^2$, а правая часть --- $a_1^2$, то уравнение будет иметь решение на
интервале $(0, -a_0)$ только при $a_0^2 > a_1^2$.

Преобразуем уравнение~\eqref{eq:D-zero}:
\begin{equation*}
  \abs{a_0 + \beta} = \abs{a_1} e^{\beta h}.
\end{equation*}
Так как $\beta \in (0, -a_0)$, то $\abs{a_0 + \beta} = - (a_0 + \beta)$,
поэтому
\begin{equation*}
  e^{\beta h} = - \frac{1}{\abs{a_1}} (a_0 + \beta).
\end{equation*}
Заметим, что $a_1 \neq 0$ по условию задачи.

Пользуясь $W$-функцией Ламберта~\cite{corless1996}, решение этого
трансцендентного уравнения представимо в виде:
\begin{equation*}
  \beta = \beta_0 = -a_0 - \frac{1}{h} W(\abs{a_1} h e^{-a_0 h}).
\end{equation*}
Если же $\beta < \beta_0$, то $D > 0$, причём
\begin{equation*}
  \begin{aligned}
    z_1 &= -(a_0 + \beta) + \sqrt{D_1}, \\
    z_2 &= -(a_0 + \beta) - \sqrt{D_1}.
  \end{aligned}
\end{equation*}
Очевидно, что $z_1 > 0$ и $z_1 > z_2$. Так как справедлива оценка
\begin{equation*}
  \sqrt{D_1}
  =
  \sqrt{
    {(a_0 + \beta)}^2 - a_1^2 e^{2 \beta h}
  }
  < \abs{a_0 + \beta}
  = -(a_0 + \beta),
\end{equation*}
то приходим к выводу, что $z_2 > 0$. Отсюда следует, что
неравенство
\begin{equation*}
  \begin{aligned}
    z^2 + 2 (a_0 + \beta) z + a_1^2 e^{2 \beta h}
    =
    (z - z_1) (z - z_2) < 0
  \end{aligned}
\end{equation*}
будет справедливо для всех $z \in (z_1, z_2)$. Так как
\begin{equation*}
  \gamma
  = \sqrt{\frac{\alpha_2}{\alpha_1}}
  = \sqrt{\frac{P + h Q}{P}}
  = \sqrt{1 + h z},
\end{equation*}
то, взяв $z = z_1 + \varepsilon$ для некоторого $\varepsilon > 0$, получим
минимальное значение для $\gamma$.

\subsection{Улучшение результата}

Пусть матрица $\mathcal{M}(P, Q) + \mathcal{N}(P)$ отрицательно
полуопределена, то есть
\begin{equation*}
  \mathcal{M}(P, Q) + \mathcal{N}(P) \preceq 0.
\end{equation*}

Воспользуемся обобщённым критерием Сильвестра~\cite[стр.~184]{ilin2014}:
\begin{equation*}
  \begin{aligned}
    \Delta_1
    &=
      2 (a_0 + \beta) P + Q \leqslant 0, \\
    \Delta_2
    &=
      -e^{-2 \beta h} Q \leqslant 0, \\
    \Delta_{1,2}
    &=
      \left[
      2 (a_0 + \beta) P + Q
      \right] e^{-2 \beta h} Q + a_1^2 P^2 \leqslant 0.
  \end{aligned}
\end{equation*}

\newpage

\addcontentsline{toc}{section}{Список использованных источников}
\renewcommand{\refname}{Список использованных источников}
\begin{center}
  \printbibliography{}
\end{center}

\end{document}
